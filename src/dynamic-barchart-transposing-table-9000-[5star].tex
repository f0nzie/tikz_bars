% create bar chart from select table row(s) for select column
% https://tex.stackexchange.com/a/355926/173708

% used PGFPlots v1.14
\documentclass[border=5pt]{standalone}
\usepackage{pgfplots,pgfplotstable}
    \pgfplotsset{
        % use this `compat' level or higher to use the advanced features for
        % axis label positioning
        compat=1.13,
    }
    \pgfplotstableread[col sep=comma]{
        Year,   A,      B,    C,   D,     E
        2011, 14.80,  9.50, 2.27, 1.13, -0.15
        2012, 15.80, 10.50, 2.57, 2.13, -0.25
        2013, 16.80, 11.50, 2.67, 3.13, -0.50
    }\datatable

    % do have a simple solution, transpose the data table
    \pgfplotstabletranspose[
        colnames from={Year},
    ]{\transposeddatatable}{\datatable}

    
    % =========================================================================
    % here you specify the year you want to plot
    \pgfmathtruncatemacro{\Year}{2013}
    % =========================================================================
    % now we can extract the `ymax' value from the transposed table
    \pgfplotstablegetelem{0}{\Year}\of{\transposeddatatable}
        \pgfmathsetmacro{\ymax}{\pgfplotsretval}
        
\begin{document}
	
\begin{tikzpicture}
    \begin{axis}[
        x=1.5cm,
        ybar,
        bar width=25pt,
        bar shift=0pt,  % offset of the bar; could be negative
        ylabel=\Year,   % apply the stored value here for the chosen year
        ymin=-0.1,      
        ymax=\ymax,     % same here. supply the stored value for the `ymax' value
        enlarge x limits=0.2,
        %
        % instead of using `symbolic x coords' we do it a bit different,
        % because then we don't get into trouble "deleting"/skipping the "A"
        % value from the table, which should not be plotted here
        %
        % just use integers to a sufficiently high number so that all bars
        % will have a tick ...
        xtick=data,
        %
        % ... and use as labels the values from the transposed table
        xticklabels from table={\transposeddatatable}{colnames},
        %
        % to skip the "A" value filter it away
        % x filter/.expression={x==0 ? NaN : x},  % enable to filter column A
        %
        % this is used to filter the "negative" values away
        % (this also works for filtering the "positive" values away, since the
        %  filter is applied *after* `y expr' is evaluated, so the former
        %  former positive values are now negative and vice versa)
        % y filter/.expression={y<0 ? NaN : y},  % enable to filter negative values
        %
        nodes near coords,  % column labels on top of bars
        tick pos=left,      % y-axis on left of plot
    ]
    	% plot only one series, which is Year 2013
        \addplot [
            fill=blue!50,
        ] table [
            % we use `\coordindex' as x value, which now matches the `xtick's
            % and the corresponding labels
            x expr=\coordindex,
            %
            % of course here we want to simply the values from the corresponding
            % chosen year
            y=\Year,
        ] {\transposeddatatable};

		% add a second layer on top of the first plot
        \addplot [
            fill=red!50,
        ] table [
            x expr=\coordindex,
            %
            % here we also want to use the values from the corresponding year,
            % but we want to plot the negative of that value why we use
            % `y expr' here and thus need `\thisrow'
            y expr=-\thisrow{\Year},
        ] {\transposeddatatable};

    \end{axis}
\end{tikzpicture}
%
%    % print tables for debugging purposes only
    % \pgfplotstabletypeset[string type]{\transposeddatatable}  % print transposed table
    % \pgfplotstabletypeset[string type]{\datatable}            % print original table
\end{document}
