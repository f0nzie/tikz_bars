% https://github.com/MartinThoma/LaTeX-examples/blob/master/tikz/bar-chart-military-budget/bar-chart-military-budget.tex
\documentclass{article}
\usepackage[pdftex,active,tightpage]{preview}
\setlength\PreviewBorder{2mm}

\usepackage{pgfplots}
\usepackage{tikz}
\usepackage{helvet}
\usepackage[eulergreek]{sansmath}

% Sans serif fonts do not look as nice as serif fonts,
% but they are much easier to read when the image is small
\pgfplotsset{
  tick label style = {font=\sansmath\sffamily},
  every axis label = {font=\sansmath\sffamily},
  legend style = {font=\sansmath\sffamily},
  label style = {font=\sansmath\sffamily}
}

\begin{document}
\begin{preview}
\begin{tikzpicture}
  \begin{axis}[
    title              = US military budget,
    % xlabel             = Year,
    ylabel             = In constant (2010) billion US\$,
    % Display option
    width  = 12cm,
    height = 10cm,
    ymin   = 340,
    ymax   = 760,
    xmin   = 1987,
    xmax   = 2011.9,
    % Configuration
    title style                         = {font=\bfseries},
    x tick label style                  = {/pgf/number format/1000 sep=,
    inner sep=0pt,
            anchor=north east,
            rotate=45 },
    legend style                        = {at={(0.5,-0.15)},
    anchor                              = north,legend columns=-1},
    ybar                                = 5pt,  % configures ‘bar shift’
    bar width                           = 9pt,
    nodes near coords,
    every node near coord/.append style = {rotate=90, anchor=west, font=\small\sansmath\sffamily},
    point meta                          = y,  % the displayed number
  ]
    \addplot
    coordinates {(1988, 540.415) (1989, 534.906) (1990, 510.998)
        (1991, 448.806) (1992, 474.215) (1993, 449.281) (1994, 421.917)
        (1995, 399.043) (1996, 377.342) (1997, 375.375) (1998, 366.918)
        (1999, 367.822) (2000, 382.061) (2001, 385.142) (2002, 432.452)
        (2003, 492.200) (2004, 536.459) (2005, 562.039) (2006, 570.769)
        (2007, 585.749) (2008, 629.095) (2009, 679.574) (2010, 698.281)
        (2011, 689.591) };
  \end{axis}
\end{tikzpicture}
\end{preview}
\end{document}
